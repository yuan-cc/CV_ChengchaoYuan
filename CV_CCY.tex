\documentclass[11pt,english]{article}
\IfFileExists{lmodern.sty}{\usepackage{lmodern}}{}
\usepackage[T1]{fontenc}
\usepackage[latin9]{inputenc}
\usepackage{babel}
\usepackage{scrextend}
\usepackage{wasysym}
\usepackage{fontawesome5}

\usepackage{academicons}
% Uncomment any of the following four lines to get different fonts
%\usepackage[garamondx,bigdelims]{newtxmath}
%\usepackage{garamondx}
%\usepackage{mathpazo}
%\usepackage{mathptmx}
\usepackage{newcent}

% Colors: see  http://www.math.umbc.edu/~rouben/beamer/quickstart-Z-H-25.html
\usepackage{color}
\usepackage[dvipsnames]{xcolor}
\definecolor{oucrimson}   {RGB}{132.,22. ,23. }
\definecolor{byublue}     {RGB}{0.  ,30. ,76. }
\definecolor{navyblue}    {RGB}{0.  ,0.  ,128.}
\definecolor{darkblue}    {RGB}{0.  ,0.  ,139.}
\definecolor{dukeblue}    {RGB}{0.  ,0.  ,156.}

% Layout
\usepackage{multirow}
\usepackage{setspace} %singlespacing; onehalfspacing; doublespacing; setstretch{1.1}
%\setstretch{1.5}
\usepackage[verbose,margin=1in]{geometry} % Margins
\setlength{\headheight}{15pt} % Sufficent room for headers
\usepackage[bottom]{footmisc} % Forces footnotes on bottom

% Headers/Footers
%\usepackage{datenumber}
\usepackage{lastpage}
\setlength{\headheight}{0pt}
\usepackage{fancyhdr}
\pagestyle{fancy}
\fancyhf{}\renewcommand{\headrulewidth}{0pt}
\lhead{} \chead{} \rhead{}
\lfoot{\today} \cfoot{}
\rfoot{Page \thepage \,of \pageref{LastPage}}

% Useful Packages
%\usepackage{bookmark} % For speedier bookmarks
%\usepackage{amsthm}   % For detailed theorems
%\usepackage{amssymb}  % For fancy math symbols
%\usepackage{amsmath}  % For awesome equations/equation arrays
\usepackage{array}    % For tubular tables
\usepackage{booktabs}
\usepackage{longtable}% For long tables
\usepackage[flushleft]{threeparttable} % For three-part tables
\usepackage{lipsum}
\setlength{\LTpre}{-5pt}
\setlength{\LTpost}{-5pt}
\usepackage{multicol} % For multi-column cells
\usepackage{graphicx} % For shiny pictures
\usepackage{subfig}   % For sub-shiny pictures
\usepackage{enumerate}% For cusomtizable lists
\usepackage{enumitem}

%fontfamily
%\usepackage{utopia}
%\usepackage{tgheros}

%\usepackage[scaled]{helvet}
%\renewcommand\familydefault{\sfdefault} 
%\usepackage[T1]{fontenc}
\usepackage{palatino}

\usepackage{pstricks,pst-node,pst-tree,pst-plot} % For trees

\setlist[enumerate]{topsep=0pt,itemsep=-1ex,partopsep=1ex,parsep=1ex}


% Bib
\usepackage[authoryear]{natbib} % Bibliography
\usepackage{url}                % Allows urls in bib

% TOC
\setcounter{tocdepth}{4}

% Links
\usepackage{hyperref}    % Always add hyperref (almost) last
\hypersetup{colorlinks,breaklinks,citecolor=black,filecolor=black,linkcolor=Cerulean,urlcolor=Cerulean}

\usepackage[all]{hypcap} % Links point to top of image, builds on hyperref
%\usepackage{breakurl}    % Allows urls to wrap, including hyperref


% Custom commands
\newcommand{\makeheading}[2]%
        {\hspace*{-\marginparsep minus \marginparwidth}%
         \begin{minipage}[t]{\textwidth\marginparwidth\marginparsep}%
         {\Huge\bfseries #1} \hfill  {\LARGE\bfseries #2 \hspace*{-2.2\marginparsep minus \marginparwidth}}\\[-0.2\baselineskip]%
                \color{Cerulean} \rule{\textwidth}{1.5pt}\rule{\marginparsep}{1.5pt}\rule{\marginparwidth}{1.5pt}%
         \end{minipage}}
         
\setlength{\parindent}{0in}
\reversemarginpar
\renewcommand{\section}[2]{
        \pagebreak[3]%
        \vspace{0.75\baselineskip}%
        \phantomsection\addcontentsline{toc}{section}{#1}%
        \hspace{0in}%
        %\marginpar{\raggedright \scshape #1}#2}
        %\rule[3.5pt]{0.9in}{1.5pt} ~~
        {\raggedright \scshape \Large \textbf{#1}}%
        \vspace{0.25\baselineskip}#2
}





\newcommand*\fixendlist[1]{%
    \expandafter\let\csname preFixEndListend#1\expandafter\endcsname\csname end#1\endcsname
    \expandafter\def\csname end#1\endcsname{\csname preFixEndListend#1\endcsname\vspace{-0.6\baselineskip}}}

\let\originalItem\item
\newcommand*\fixouterlist[1]{%
    \expandafter\let\csname preFixOuterList#1\expandafter\endcsname\csname #1\endcsname
    \expandafter\def\csname #1\endcsname{\let\oldItem\item\def\item{\pagebreak[2]\oldItem}\csname preFixOuterList#1\endcsname}
    \expandafter\let\csname preFixOuterListend#1\expandafter\endcsname\csname end#1\endcsname
    \expandafter\def\csname end#1\endcsname{\let\item\oldItem\csname preFixOuterListend#1\endcsname}}
\newcommand*\fixinnerlist[1]{%
    \expandafter\let\csname preFixInnerList#1\expandafter\endcsname\csname #1\endcsname
    \expandafter\def\csname #1\endcsname{\let\oldItem\item\let\item\originalItem\csname preFixInnerList#1\endcsname}
    \expandafter\let\csname preFixInnerListend#1\expandafter\endcsname\csname end#1\endcsname
    \expandafter\def\csname end#1\endcsname{\csname preFixInnerListend#1\endcsname\let\item\oldItem}}

\newlist{outerlist}{itemize}{3}
    \setlist[outerlist]{label=\enskip\textbullet,leftmargin=*}
    \fixendlist{outerlist}
    \fixouterlist{outerlist}

\newlist{lonelist}{itemize}{3}
    \setlist[lonelist]{label=\enskip\textbullet,leftmargin=*,partopsep=0pt,topsep=0pt}
    \fixendlist{lonelist}
    \fixouterlist{lonelist}

\newlist{innerlist}{itemize}{3}
    \setlist[innerlist]{label=\enskip\textbullet,leftmargin=*,parsep=0pt,itemsep=0pt,topsep=0pt,partopsep=0pt}
    \fixinnerlist{innerlist}

\newlist{loneinnerlist}{itemize}{3}
    \setlist[loneinnerlist]{label=\enskip\textbullet,leftmargin=*,parsep=0pt,itemsep=0pt,topsep=0pt,partopsep=0pt}
    \fixendlist{loneinnerlist}
    \fixinnerlist{loneinnerlist}



\begin{document}
%\thispagestyle{empty}
\makeheading{\color{Cerulean}Chengchao Yuan}{\href{https://scholar.google.com/citations?user=esUZFoMAAAAJ&hl=en}{\aiGoogleScholarSquare} \href{https://orcid.org/my-orcid?orcid=0000-0003-0327-6136}{\aiOrcid} \href{https://ui.adsabs.harvard.edu/public-libraries/NCRLXpiDTnGg2zwnvpAzRw}{\aiADSSquare} \href{https://inspirehep.net/authors/1671091}{\aiInspireSquare} \href{https://github.com/yuan-cc}{\faGithubSquare} \href{https://www.facebook.com/profile.php?id=100009428177031}{\faFacebookSquare} \href{https://twitter.com/YuanChengchao}{\faTwitterSquare}}
\vspace{-0.7em}
% ADDRESS/CONTACT HEADER
% ==================================================================================
\setstretch{1.0}
\begin{multicols}{2}
 \hypersetup{colorlinks,breaklinks,citecolor=black,filecolor=black,linkcolor=oucrimson,urlcolor=oucrimson}

\begin{flushleft}
Department of Physics

322 Osmond Lab, University Park

PA 16802, USA

\end{flushleft}
       

\begin{flushright}
\textit{\faPhoneSquare} +1 (814)954-2785

\textit{\faEnvelope} \href{mailto:cxy52@psu.edu}{cxy52@psu.edu}

\textit{\faLink} \href{https://yuan-cc.github.io}{yuan-cc.github.io}

\end{flushright}

\end{multicols}
% ==================================================================================


 \hypersetup{colorlinks,breaklinks,citecolor=black,filecolor=black,linkcolor=oucrimson,urlcolor=oucrimson}

\setstretch{1.2}
\vspace{-.25in}
\section{\color{Cerulean}Research Interests}

\vspace{-17pt}
{\rule[-0.5ex]{\columnwidth}{0.4pt}}
\begin{innerlist}
\item High-energy astrophysics (particle acceleration, transport and radiation processes) 
\item Multimessenger astrophysics (gamma rays, neutrinos and cosmic rays from extreme sources)
\end{innerlist}
%I am working on various aspects of multi-messenger astrophysics, including the origins and implications of high-energy photons, cosmic rays and neutrinos from individual cosmic sources and source populations. In particular, I model the {\bf acceleration, transport and radiation processes} of high-energy particles originated from {\bf compact binary pairs (super massive/stellar mass black holes), supernovae, gamma-ray bursts, active galactic nuclei and galaxy/cluster mergers}. I am interested in how the multi-messenger analyses enlighten our understanding on the physical nature of high-energy astrophysical phenomena.
\vspace{-5pt}
\section{\color{Cerulean}Education}

\vspace{-17pt}
\rule[-0.5ex]{\columnwidth}{0.4pt}
% ==================================================================================
\begin{tabular}{p{1.05in}>{\hangindent=1em}p{5.45in}<{\raggedright}}
 08/2022& \bf Ph.D. in Physics, Pennsylvania State University \\
   & \small{\ Supervised by Prof. \href{http://personal.psu.edu/nnp/}{P\'eter M\'esz\'aros} and Prof. \href{https://science.psu.edu/physics/people/kohta-murase}{Kohta Murase} } \\
     & \small{\ Thesis: \it  Neutrino and Electromagnetic Counterparts of Galaxy and Astrophysical Black Hole Mergers} \\
06/2016 & \bf B.Sc. in Astronomy, Nanjing University, China \\
& \small{\ Supervised by Prof. Xiangyu Wang and Prof. Fayin Wang}\\
& \small{\ Undergraduate Thesis: \it The origin of high-energy astrophysical neutrinos}
\end{tabular}

\section{\color{Cerulean}Employment History}

\vspace{-17pt}
\rule[-0.5ex]{\columnwidth}{0.4pt}
\begin{tabular}{p{1.05in}>{\hangindent=0.5em}p{5.65in}<{\raggedright}}
2018 - 2022 & \textbf{Research Assistant}, Dept. of Physics, Penn State\\
2016 - 2022& \textbf{Teaching Assistant}, Dept. of Physics, Penn State\\
Summer 2015 & \textbf{REU Intern}, Dept. of Astronomy and Astrophysics, Penn State
\end{tabular}

\section{\color{Cerulean}Honors \& Awards}

\vspace{-17pt}
\rule[-0.5ex]{\columnwidth}{0.4pt}
\begin{tabular}{p{.85in}>{\hangindent=1em}p{5.65in}<{\raggedright}}
2022 & {\bf TDLI Prize Postdoctoral Fellowship}, Tsung-Dao Lee Institute (declined)\\
2022, 21 &  {\bf W. Donald Miller Graduate Fellowship}, Pennsylvania State University\\
2019-2022 &  {\bf David C. Duncan Graduate Fellowship}, Pennsylvania State University \\
2018 &  {\bf APS Graduate Student Travel Grant}, American Physical Society \\
2017 &  {\bf Homer F. Braddock Scholarship}, Pennsylvania State University\\
2016 & {\bf School of Astronomy and Space Science Dean's Scholarship}, Nanjing University\\
2016 & {\bf Outstanding Thesis Award,} Nanjing University\\
2015 & {\bf REU Intern Travel Grant }(host institution: Penn State), Nanjing University
\end{tabular}
\newcommand*{\doii}[1]{\href{http://dx.doi.org/#1}{doi: #1}}

\vspace{-7pt}
\section{\color{Cerulean}Publications}

\vspace{-17pt}
\rule[-0.5ex]{\columnwidth}{0.4pt}

\begin{addmargin}[0.56em]{2em}
{\bf \small{Journal articles (first-author: 7)}}
\end{addmargin}
\begin{enumerate}

\item[{[8]}] {\bf Yuan, C.}, Murase, K., Guetta, D., Pe'er, A., Bartos, I., \&  M\'esz\'aros, P., (2021) ``GeV Signature of Short Gamma-Ray Bursts in Active Galactic Nuclei", \href{https://arxiv.org/abs/2112.07653}{arXiv: 2112.07653}

\item[{[7]}] {{\bf Yuan, C.}, Murase, K., Zhang, B. T., Kimura, S. S. \& M\'esz\'aros, P. (2021) ``Post-Merger Jets from Supermassive Black Hole Coalescences as Electromagnetic Counterparts of Gravitational Wave Emission'', \emph{ApJL}, 911 L15,  \doii{10.3847/2041-8213/abee24}}

\item[{[6]}] {Zhang, T. B., Murase, K., {\bf Yuan, C}., Kimura, S. S. \& M\'esz\'aros, P. (2020) ``External Inverse-Compton Emission Associated with Extended and Plateau Emission of Short Gamma-Ray Bursts: Application to GRB 160821B'', \emph{ApJL} 908 L36, \doii{10.3847/2041-8213/abe0b0}}

\item[{[5]}] {{\bf Yuan, C.}, Murase, K., Kimura, S. \& M\'esz\'aros, P. (2020) ``High-energy neutrino emission subsequent to gravitational wave radiation from supermassive black hole mergers'', \emph{Phys. Rev. D} 102, 083013. \doii{10.1103/PhysRevD.102.083013}}

\item[{[4]}] {{\bf Yuan, C.}, Murase, K. \& M\'esz\'aros, P. (2020) ``Complementarity of Stacking and Multiplet Constraints on the Blazar Contribution to the Cumulative High-Energy Neutrino Intensity'', \emph{ApJ}, 890:1. \doii{10.3847/1538-4357/ab65ea}}

\item[{[3]}] {{\bf Yuan, C.}, Murase, K. \& M\'esz\'aros, P. (2019) ``Secondary Radio and X-ray Emissions from Galaxy Mergers'', \emph{ApJ}, 878:76. \doii{10.3847/1538-4357/ab1f06}}

\item[{[2]}] {{\bf Yuan, C.}, M\'esz\'aros, P., Murase K. \& Jeong, D. (2018) ``Cumulative Neutrino and Gamma-Ray Backgrounds from Halo and Galaxy Mergers'', \emph{ApJ}, 857:50. \doii{10.3847/1538-4357/aab774}}

\item[{[1]}] {{\bf Yuan, C.} \& Wang, F. (2015) ``Cosmological Test Using Strong Gravitational Lensing Systems'', \emph{MNRAS}, 452:3. \doii{10.1093/mnras/stv1444}}
\end{enumerate}


\begin{addmargin}[0.56em]{2em}
{\bf \small{ Conference proceedings and other articles}}
\end{addmargin}
\begin{enumerate}

\item[{[2]}] {{\bf Yuan, C.}, M\'esz\'aros, P., Murase K. \& Jeong, D. (2018) ``Cumulative Neutrino and Gamma-Ray Backgrounds from Halo and Galaxy Mergers'', in \emph{APS April meeting: U17.004}.  \href{https://ui.adsabs.harvard.edu/abs/2018APS..APRU17004Y}{{Talk abstract}}}
\item[{[1]}] {{\bf Yuan, C.}, Murase K. \& M\'esz\'aros, P. (2019) ``A Multi-Messenger Picture of Galaxy Mergers: Neutrinos and Electromagnetic Emissions'', \emph{(ICRC2019) 1041}. \href{https://pos.sissa.it/358/1041/pdf}{{Proceedings of Science}}}

\end{enumerate}
\vspace{-7pt}
\section{\color{Cerulean}Conferences and Scientific Talks}

\vspace{-17pt}
\rule[-0.5ex]{\columnwidth}{0.4pt}
\vspace{-5pt}
% ==================================================================================
\begin{longtable}{p{.85in}>{\hangindent=1em}p{5.65in}<{\raggedright}}
03/2022 & {\bf Seminar talk:} University of Maryland, virtual\\
12/2021 & {\bf Seminar talk:} Columbia University, in-person\\
11/2021 & {\bf Seminar talk:} DESY, virtual\\
10/2021 & {\bf Seminar talk:} UNLV, virtual\\
07/2021 & {\bf Contributed talk:} EPS Conference on High Energy Physics, virtual\\
04/2021 & {\bf Contributed talk: }APS April meeting, virtual\\
02/2021 & {\bf Lunch talk: }Institute for Gravitation and the Cosmos (IGC), Penn State, virtual\\
10/2020 & {\bf Invited talk: }CCAPP AstroParticle Lunch, Ohio State University, virtual\\
10/2020 &  {\bf Seminar talk: } Tohoku University, Japan, virtual\\
09/2020 & {\bf Lunch talk: }Dept. of Astronomy \& Astrophysics, Penn State University, virtual\\
08/2020 &{\bf Invited talk: }{Time-Domain High-Energy Messenger Astrophysics Workshop}, University of Kyoto, Japan, virtual\\
07/2019 &{\bf Poster: }{36th International Cosmic Ray Conference (ICRC)}, Madison, WI\\
06/2019 &{\bf Contributed talk: }{IGC@25: Multimessenger Universe Workshop}, State College, PA\\
04/2018 &{\bf Contributed talk: }{APS April meeting}, Columbus, OH\\
08/2015 & {\bf Lunch talk: }Dept. of Astronomy \& Astrophysics, Penn State University\\
\end{longtable}
\section{\color{Cerulean}Code Development}

\vspace{-17pt}
\rule[-0.5ex]{\columnwidth}{0.4pt}
\begin{addmargin}[1em]{2em}
{\bf Astrophysical Multimessenger Emission Synthesizer (AMES)}\\
\emph{A time-dependent numerical code for the production and propagation of high-energy cosmic rays, neutrinos, and gamma-rays for various astrophysical environments}
\begin{innerlist}
\item Developed the code for photo-meson/photo-hadronic interaction cross sections and cosmic $\gamma\gamma$ interactions.
\end{innerlist}
\end{addmargin}
% ==================================================================================

\section{\color{Cerulean}Programming Skills}

\vspace{-17pt}
\rule[-0.5ex]{\columnwidth}{0.4pt}
\begin{innerlist}
\item Extensive experience in using {\bf\texttt{CRpropa}}, an astrophysical simulation code for the propagation of ultra-high-energy particles.
\item Programming languages:  C++, Python, Mathematica and Fortran
\end{innerlist}

\section{\color{Cerulean}Teaching Experience}

\vspace{-17pt}
\rule[-0.5ex]{\columnwidth}{0.4pt}
\begin{tabular}{p{1.45in}>{\hangindent=1em}p{5.35in}<{\raggedright}}
2021 Fall (F)& T.A. PHYS 561: Quantum Mechanics\\
2021 Spring (S)  & T.A. PHYS 400: Electrodynamics\\
2020 F & T.A. PHYS/MATH 479: Special and General Relativity\\
2018 S, 2019, 2020 S & Lab. T.A. PHYS 250: Introductory Physics\\
2018 F & T.A. PHYS 525: Methods of Theoretical Physics \\
2016 F - 2017 F, 2022 S &Lab. T.A. PHYS 212: Electromagnetism 
\end{tabular}

\section{\color{Cerulean}Selected Professional/Outreach Experience}

\vspace{-17pt}
\rule[-0.5ex]{\columnwidth}{0.4pt}
\begin{tabular}{p{1.15in}>{\hangindent=1em}p{5.25in}<{\raggedright}}
2021 & {\bf Abstract Sorting Committee} of AAS 239th Annual Meeting \\
2021 & {\bf Journal Club Organizer} for the Center of Multimessenger Astrophysics\\
2017 - 2022 & {\bf Guest Lecturer} and {\bf A Tour of Universe Demonstrator} at AstroFest (4-night outreach, 2500+ public
visitors)\\ 
2018 & {\bf Astropy Demonstrator} at
K-12 Educators - Bring Cutting-Edge STEM Research into your Classroom (2-day outreach, 100+ high-school teachers)
\end{tabular}


\section{\color{Cerulean}References}

\vspace{-17pt}
\rule[-0.5ex]{\columnwidth}{0.4pt}

\begin{addmargin}[1em]{2em}
\textbf{Dr. P\'eter M\'esz\'aros} (\href{mailto:nnp@psu.edu}{nnp@psu.edu})
\end{addmargin}
\begin{addmargin}[2em]{2em}
 Eberly Chair Professor, Astronomy \& Astrophysics and Physics\\
 Pennsylvania State University, USA
\end{addmargin}


\begin{addmargin}[1em]{2em}
\textbf{Dr. Kohta Murase} (\href{mailto:murase@psu.edu}{murase@psu.edu})
\end{addmargin}
\begin{addmargin}[2em]{2em}
 Associate Professor, Physics and Astronomy \& Astrophysics\\
 Pennsylvania State University, USA
\end{addmargin}


\begin{addmargin}[1em]{2em}
\textbf{Dr. Dafne Guetta} (\href{mailto:dafneguetta@braude.ac.il}{dafneguetta@braude.ac.il})
\end{addmargin}
\begin{addmargin}[2em]{2em}
Professor of Physics, ORT Braude College, Israel
\end{addmargin}


\begin{addmargin}[1em]{2em}
\textbf{Dr. Donghui Jeong} (\href{mailto:djeong@psu.edu}{djeong@psu.edu})
\end{addmargin}
\begin{addmargin}[2em]{2em}
 Associate Professor, Astronomy \& Astrophysics\\
 Pennsylvania State University, USA
 \end{addmargin}


%\begin{addmargin}[1em]{2em}
%\textbf{Dr. Xiangyu Wang} (\href{mailto:xywang@nju.edu.cn}{xywang@nju.edu.cn})
%\end{addmargin}
%\begin{addmargin}[2em]{2em}
%Professor, Astronomy \& Space Science, Nanjing University, China
%\end{addmargin}





\end{document}

%%%%%%%%%%%%%%%%%%%%%%%%%% End CV Document %%%%%%%%%%%%%%%%%%%%%%%%%%%%%
